% Experimental Results Section

\section{Experimental Results}
\label{sec:experimental_results}

% Main results
\subsection{Main Results}
\label{subsec:main_results}

Present your main experimental results. Reference tables and figures as needed.

% Example results table
\begin{table}[htbp]
    \centering
    \caption{Comparison of methods on benchmark datasets. Best results are in \textbf{bold}.}
    \label{tab:main_results}
    \begin{tabular}{lcccc}
        \toprule
        \textbf{Method} & \textbf{Dataset 1} & \textbf{Dataset 2} & \textbf{Dataset 3} & \textbf{Average} \\
        \midrule
        Baseline 1 & 85.2 & 78.3 & 72.1 & 78.5 \\
        Baseline 2 & 86.7 & 79.8 & 73.4 & 79.9 \\
        Ours & \textbf{89.3} & \textbf{82.1} & \textbf{76.8} & \textbf{82.7} \\
        \bottomrule
    \end{tabular}
\end{table}

% Example figure reference
An example visualization of the results is shown in \Cref{fig:example}.

\begin{figure}[htbp]
    \centering
    % \includegraphics[width=0.8\textwidth]{example_figure}
    \fbox{\parbox{0.8\textwidth}{\centering\vspace{2cm}Placeholder for figure\vspace{2cm}}}
    \caption{Description of the figure showing the main results.}
    \label{fig:example}
\end{figure}

% Ablation study
\subsection{Ablation Study}
\label{subsec:ablation}

Present ablation studies to analyze the contribution of different components.

% Additional analysis
\subsection{Additional Analysis}
\label{subsec:additional_analysis}

Include any additional analysis, such as:
\begin{itemize}
    \item Sensitivity analysis
    \item Computational efficiency
    \item Qualitative results
\end{itemize}
